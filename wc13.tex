\documentclass[a4paper, addpoints]{exam}

\usepackage{amsfonts,amsmath,amsthm}
\usepackage{framed}
\usepackage[a4paper]{geometry}

\header{CS/MATH 113}{WC13: Structural Induction}{Spring 2024}
\footer{}{Page \thepage\ of \numpages}{}
\runningheadrule
\runningfootrule

\printanswers

\qformat{{\large\bf \thequestion. \thequestiontitle}\hfill}
\boxedpoints

\theoremstyle{definition}
\newtheorem{definition}{Definition}
\theoremstyle{claim}
\newtheorem{claim}{Claim}

\title{Weekly Challenge 13: Structural Induction}
\author{CS/MATH 113 Discrete Mathematics}
\date{Spring 2024}

\begin{document}
\maketitle

\begin{questions}
\titledquestion{$k$-ary tree}[10]
  Definition 5 in Section 5.3 of our textbook defines a \textit{full binary tree}. We extend this definition to a \textit{full $k$-ary tree} as follows.
  \begin{framed}
    \begin{definition}[Full $k$-ary tree]$\null$
      
      \underline{Basis Step} There is a full $k$-ary tree consisting only of a single vertex $r$.
      
      \underline{Recursive Step}  If $T_1,T_2, T_3,\ldots,T_k$ are disjoint full $k$-ary trees, there is a full $k$-ary tree, denoted by $T_1\cdot T_2\cdot T_3\cdot\ldots\cdot T_k$, consisting of a root $r$ together with edges connecting the root to each of the roots of $T_1,T_2, T_3,\ldots,T_k$.
    \end{definition}
  \end{framed}
  We also introduce the following definitions of nodes in a tree.
  \begin{definition}[Leaf node]
    A leaf node in a tree is a node that has no children.
  \end{definition}
  \begin{definition}[Internal node]
    An internal node in a tree is a node that is not a leaf node.
  \end{definition}

  Use structural induction to prove the following claim.
  \begin{claim}
    The number of internal nodes in a full $k$-ary tree with $n$ leaves is $\frac{n-1}{k-1}$.
  \end{claim}
  \begin{solution}
    % Enter your solution here.
    \renewcommand\qedsymbol{$\square$}
    \begin{proof}
      Let I(n) be the number of internal nodes in a full $k$-ary tree with $n$ leaves.

      \textit{Base Case:} When $n=1$, the full $k$-ary tree consists of a single vertex $r$ which is both the root and the leaf. The number of internal nodes is $\frac{1-1}{k-1}=0$.
      This is clearly the case since there are no internal nodes in a single-vertex tree.

      \textit{Inductive Hypothesis:} Assume that the claim holds for all full $k$-ary trees till $n$ leaves.

      \textit{Inductive Step:} We will show that the claim holds for a full $k$-ary tree with $n+1$ leaves. \\
      Let $T_k$ be a full $k$-ary tree and the sum of the number of leaves in $T_k$ be $n+1$. \\
      By the definition of a $k$-ary three, $T_k$ can be represented as $T_1\cdot T_2\cdot T_3\cdot\ldots\cdot T_{k-1}$ \\
      Then the internal nodes in $T_k$ can be written as: \\
      $I(n+1) \equiv \sum_{i}^{k-1} \frac{n_i - 1}{k -1}$, where $1 \leq i \leq (k-1)$ \\
      This is beacuse the number of internal nodes in $T_k$ is the sum of the number of internal nodes in each of the $k-1$ subtrees. \\
      Since $n_{k-1} \leq n$ we can apply the inductive hypothesis to each of the $k-1$ subtrees. \\

      Since, $I(n+1) \equiv \sum_{i}^{k-1} n_i \equiv n+1$, where $1 \leq i \leq (k-1)$ \\
      Finally we get the equation: $\frac{n}{k -1}$ \\

      This matches the equation that we get when $n+1$ is plugged into the formula for $I(n)$ i.e. $\frac{n}{k -1}$ \\

      Therefore, the claim holds for a full $k$-ary tree with $m+1$ leaves. \\
      By the principle of structural induction, the claim holds for all full $k$-ary trees with $n$ leaves, where $n\geq 1$.
    \end{proof}
  \end{solution}
\end{questions}

\end{document}
%%% Local Variables:
%%% mode: latex
%%% TeX-master: t
%%% End:
