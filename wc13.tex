\documentclass[a4paper, addpoints]{exam}

\usepackage{amsfonts,amsmath,amsthm}
\usepackage{framed}
\usepackage[a4paper]{geometry}

\header{CS/MATH 113}{WC13: Structural Induction}{Spring 2024}
\footer{}{Page \thepage\ of \numpages}{}
\runningheadrule
\runningfootrule

\printanswers

\qformat{{\large\bf \thequestion. \thequestiontitle}\hfill}
\boxedpoints

\theoremstyle{definition}
\newtheorem{definition}{Definition}
\theoremstyle{claim}
\newtheorem{claim}{Claim}

\title{Weekly Challenge 13: Structural Induction}
\author{CS/MATH 113 Discrete Mathematics}
\date{Spring 2024}

\begin{document}
\maketitle

\begin{questions}
\titledquestion{$k$-ary tree}[10]
  Definition 5 in Section 5.3 of our textbook defines a \textit{full binary tree}. We extend this definition to a \textit{full $k$-ary tree} as follows.
  \begin{framed}
    \begin{definition}[Full $k$-ary tree]$\null$
      
      \underline{Basis Step} There is a full $k$-ary tree consisting only of a single vertex $r$.
      
      \underline{Recursive Step}  If $T_1,T_2, T_3,\ldots,T_k$ are disjoint full $k$-ary trees, there is a full $k$-ary tree, denoted by $T_1\cdot T_2\cdot T_3\cdot\ldots\cdot T_k$, consisting of a root $r$ together with edges connecting the root to each of the roots of $T_1,T_2, T_3,\ldots,T_k$.
    \end{definition}
  \end{framed}
  We also introduce the following definitions of nodes in a tree.
  \begin{definition}[Leaf node]
    A leaf node in a tree is a node that has no children.
  \end{definition}
  \begin{definition}[Internal node]
    An internal node in a tree is a node that is not a leaf node.
  \end{definition}

  Use structural induction to prove the following claim.
  \begin{claim}
    The number of internal nodes in a full $k$-ary tree with $n$ leaves is $\frac{n-1}{k-1}$.
  \end{claim}
  \begin{solution}
    % Enter your solution here.
    \renewcommand\qedsymbol{$\square$}
    \begin{proof}
      We will prove the claim by structural induction on the number of leaves $n$ in a full $k$-ary tree.
      
      \textit{Base Case:} When $n=1$, the full $k$-ary tree consists of a single vertex $r$ which is both the root and the leaf. The number of internal nodes is $\frac{1-1}{k-1}=0$.

      \textit{Inductive Hypothesis:} Assume that the claim holds for all full $k$-ary trees with $n$ leaves, where $1\leq n\leq m$ for some $m\geq 1$.

      \textit{Inductive Step:} We will show that the claim holds for a full $k$-ary tree with $m+1$ leaves.
      Let $T_1,T_2,\ldots,T_k$ be disjoint full $k$-ary trees with $n_1,n_2,\ldots,n_k$ leaves, respectively, such that $n_1+n_2+\cdots+n_k=m+1$.
      By the inductive hypothesis, the number of internal nodes in $T_i$ is $\frac{n_i-1}{k-1}$ for $1\leq i\leq k$.
      The total number of internal nodes in the full $k$-ary tree $T_1\cdot T_2\cdot\ldots\cdot T_k$ is
      \begin{align*}
        \sum_{i=1}^{k}\frac{n_i-1}{k-1}&=\frac{1}{k-1}\sum_{i=1}^{k}n_i-\frac{k}{k-1}\\
        &=\frac{1}{k-1}(m+1)-\frac{k}{k-1}\\
        &=\frac{m+1}{k-1}-\frac{k}{k-1}\\
        &=\frac{m+1-k}{k-1}\\
        &=\frac{(m+1)-1}{k-1}.
      \end{align*}
      Therefore, the claim holds for a full $k$-ary tree with $m+1$ leaves. \\
      By the principle of structural induction, the claim holds for all full $k$-ary trees with $n$ leaves, where $n\geq 1$.
    \end{proof}
  \end{solution}
\end{questions}

\end{document}
%%% Local Variables:
%%% mode: latex
%%% TeX-master: t
%%% End:
